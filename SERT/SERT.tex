\documentclass[]{article}

%opening
\title{Sistemi Embedded e Real Time}
\author{Daniele Ferrarelli}
\date{}

\begin{document}

\maketitle

\section{Introduzione Sistemi Embedded}
Definizione di un sistema Embedded: Sistema programmabile che non è pensato per essere riprogrammato dall'utente, cioè per una applicazione specializzata. Ha una interazione forte con l'ambiente circostante, sistema immerso o integrato.
La sua interfaccia utente o è invisibile o comunque non conforme a quella di un normale calcolatore elettronico, per esempio display LCD o LED o proprio invisibili all'utente. 
Sono sistemi \b{Pervasivi} , sono molti e sembrano scomparire nell'ambiente. 
\end{document}
